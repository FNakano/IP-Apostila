\documentclass{article}
\usepackage{amsmath}
\begin{document}

Na aula do dia 10 de setembro, analisamos o pior caso do tempo de execução do insertion-sort. Chegamos à (fórmula de) recorrência:

$
def.:\; \begin{cases}
 T(1)=c \\ 
 T(n)=T(n-1)+c*n+c 
\end{cases} \\
$

Um colega propôs que a fórmula abaixo equivale à recorrência (com a vantagem de não ser calculada a partir dos termos anteriores), mas não conhecíamos nenhuma técnica para demonstrar que ela é válida para todo $n>0$.


$
\\
conjectura:\;T(n)=c*\big{(}2n-1+\frac{n(n-1)}{2}\big{)} \\
$

\textbf{nota}: Nas notas da aula de 10 de setembro, faltou a constante que incluí agora.

A técnica que, agora, conhecemos é a demonstração por indução. Então vamos demonstrar, por indução, que a conjectura equivale à definição.

Iniciando pelo caso base:

$
\\
base: \textrm{para } n=1 \\
\textrm{(pela def.)    } T(n)= c\\
\textrm{(pela conj.)    } T(n)= c \\
$

Como os resultados são iguais, para o caso base, definição e conjectura são equivalentes.

Escolhe-se a estratégia de demonstração do passo: \textit{Supondo que vale para n-1, demonstre que vale para n}. Desta forma, a hipótese de indução é:

$
\\
T(n-1)=c*\big{(}2(n-1)-1+\frac{(n-1)(n-2)}{2}\big{)} \\
$

A hipótese de indução pode ser usada na demonstração do passo de indução, quantas vezes for necessária e quando for conveniente.

Seguindo para a demonstração, inicia-se com alguma verdade, por exemplo, pela definição:

$
\\
T(n)=T(n-1)+c*n+c \\
$

substituindo $T(n-1)$ pela hipótese de indução e calculando, obtém-se: 

$
\\
T(n)=c*\big{(}2(n-1)-1+\frac{(n-1)(n-2)}{2}+n+1\big{)} \\
T(n)=c*\big{(}3n-2+\frac{(n-1)(n-2)}{2}\big{)} \\
$

Como a fórmula em que se quer chegar contém o termo $2n-1$, então já o separo, o que resulta:

$
\\
T(n)=c*\big{(}2n-1+(n-1)+\frac{(n-1)(n-2)}{2}\big{)} \\
$

Agora trabalho sobre o restante da expressão para chegar à fórmula:

$
\\
T(n)=c*\big{(}2n-1+\frac{2(n-1)+(n-1)(n-2)}{2}\big{)} \\
$

Vou ocupar-me somente com o numerador: 

$
\\
2(n-1)+(n-1)(n-2)=2n-2+n^2-3n+2=n^2-n=n(n-1) \\
$

Voltando aa expressao inteira:

$
\\
T(n)=c*\big{(}2n-1+\frac{n(n-1)}{2}\big{)} \\
$

Que \'{e} igual à conjectura, logo, demonstramos que a conjectura e definição são equivalentes para todo $n>0$.

\end{document}

